\documentclass[12pt,a4paper]{report}

\usepackage{alltt, fancyvrb, url}
\usepackage{graphicx}
\usepackage[utf8]{inputenc}
\usepackage{float}
\usepackage{xcolor}
\usepackage{hyperref}
\usepackage{longtable}
\usepackage{listings}
\usepackage{color}
\usepackage[utf8]{inputenc}

\usepackage{enumitem}
\usepackage{geometry}
\usepackage{pdfpages}

\geometry{margin=1in}

\definecolor{codegray}{rgb}{0.5,0.5,0.5}
\definecolor{codepurple}{rgb}{0.58,0,0.82}
\definecolor{backcolour}{rgb}{0.95,0.95,0.92}

\lstdefinestyle{sqlstyle}{
  language=SQL,
  backgroundcolor=\color{backcolour},
  commentstyle=\color{codegreen},
  keywordstyle=\color{codepurple},
  numberstyle=\numberstyle,
  stringstyle=\color{codepurple},
  basicstyle=\footnotesize\ttfamily,
  breakatwhitespace=false,
  breaklines=true,
  captionpos=b,
  keepspaces=true,
  numbers=left,
  numbersep=10pt,
  showspaces=false,
  showstringspaces=false,
  showtabs=false
}
\newcommand\numberstyle[1]{%
  \footnotesize
  \color{codegray}%
  \ttfamily
  \ifnum#1<10 0\fi#1 |%
}

\usepackage{textcomp}
\usepackage{siunitx}

\usepackage[english]{babel}
\usepackage[capitalise, english]{cleveref}

\graphicspath{ {./src/img} }

\textwidth=450pt\oddsidemargin=0pt
\begin{document}

\begin{titlepage}
  \begin{center}
    {{\Large{\textsc{Alma Mater Studiorum $\cdot$ University of
    Bologna}}}}\\[1ex]
    \rule{15.8cm}{0.6mm}\\[2ex]
    {\small Bachelor's Degree in Computer Engineering and Science}\\
    {\small{ A.A. 2025/26}}
  \end{center}

  \vfill
  \begin{center}
    {\LARGE{\bf Smart Drone Hangar}}\\[1ex]
  \end{center}
  \vfill

  \begin{center}
    \begin{tabular}{@{}c@{\qquad}c@{}}
      {\large\bf Grazia Bochdanovits de Kavna} & {\large\bf
      Alessandro Rebosio} \\[0.5ex]
      {\small matr.\ 0001117082} & {\small matr.\ 0001130557}
    \end{tabular}
  \end{center}

  \vspace{2mm}
  \noindent\rule{\linewidth}{0.4pt}

\end{titlepage}

\tableofcontents

\chapter{Analysis}

This chapter analyses the functional behaviour and hardware structure of the Smart Drone Hangar, outlining its operating sequences, safety logic, and implementation requirements.  
The objective is to define how the system reacts to flight commands and environmental conditions, and how each hardware element contributes to its autonomous operation.

\section{Description and Requirements}

At startup, the system initializes all components and reads the sensors to determine
the current position of the drone.  
Depending on the measurements from the distance sensor (DDD) and the presence sensor
(DPD), the Context is set to either \texttt{DRONE INSIDE} or \texttt{DRONE OUTSIDE}.  
Once the state is established, LED~L1 turns on if the drone is detected inside, while
LEDs~L2 and~L3 remain off, and the LCD displays the corresponding status message.

\vspace{\baselineskip}
\noindent\textbf{Take-off Sequence.}  
When a take-off command is received from the Drone Remote Unit (DRU), the hangar gate
opens and the LCD displays ``\texttt{TAKE OFF}''.  
The system then waits for the drone to exit the hangar. Exit is confirmed when the
distance measured by the DDD exceeds $D_1$ for longer than $T_1$.  
At that moment, the gate closes automatically and the LCD updates to
``\texttt{DRONE OUTSIDE}''.

\vspace{\baselineskip}
\noindent\textbf{Landing Sequence.}  
When the drone requests landing through the DRU, the presence sensor (DPD) detects
its approach. The hangar gate opens and  
landing is confirmed when the DDD measures a distance below $D_2$ for longer than
$T_2$. Once the drone is safely inside, the gate closes and the display returns to
``\texttt{DRONE INSIDE}''.

\vspace{\baselineskip}
\noindent\textbf{Indicators and Lighting.}  
During take-off and landing operations, LED~L2 blinks with a 0.5\,s period to indicate
activity. In all other states it remains off.  
LED~L1 indicates the presence of the drone inside the hangar, while LED~L3 signals
an alarm condition.

\vspace{\baselineskip}
\noindent\textbf{Temperature Monitoring and Safety Conditions.}  
Temperature control is active whenever the drone is inside the hangar.  
If the internal temperature exceeds \texttt{Temp1} for more than \texttt{T3}, the
system enters a \textbf{pre-alarm} condition: new take-offs or landings are suspended,
but ongoing operations are allowed to finish.  
If the temperature rises above \texttt{Temp2} (\texttt{Temp2 > Temp1}) for longer than
\texttt{T4}, the system switches to \textbf{alarm} mode.  
In this state, the gate closes (if open), LED~L3 turns on, and the LCD displays
``\texttt{ALARM}''.  
If the drone is outside during an alarm, the same message is sent through the DRU.  
All operations remain suspended until the \textbf{RESET} button is pressed, restoring
normal operation.

\vspace{\baselineskip}
\noindent\textbf{Configurable Parameters and GUI Requirements.}  
The parameters $D_1$, $D_2$, $T_1$, $T_2$, $T_3$, $T_4$, \texttt{Temp1}, and
\texttt{Temp2} are configurable for testing and calibration.  
The Drone Remote Unit (DRU) graphical interface allows the user to send take-off and
landing commands, monitor the current drone and hangar states (\emph{rest},
\emph{taking off}, \emph{operating}, \emph{landing}), and view real-time data such as
temperature and distance from the ground.

\section{Summary}

The Smart Drone Hangar operates as an autonomous system that controls flight access,
monitors temperature and safety conditions, and coordinates all flight operations.  
At startup, the sensors determine whether the drone is inside or outside the hangar,
ensuring that the system state always matches the physical configuration.  
During normal operation, take-off and landing are managed automatically: the gate
opens, LED~L2 blinks to indicate activity, and sensor readings confirm the drone’s
movement before closing the gate.  
Temperature thresholds ensure that flights occur only under safe conditions, while
the alarm mode provides full protection in case of overheating.  
All operational parameters can be configured and monitored through the DRU interface,
resulting in a reliable, modular, and self-contained control system.

\newpage
\section{Wiring}
Table~\ref{tab:hardware} lists the essential hardware components required
to build and test the Smart Drone Hangar. Quantities, component types, and
their function are summarized for procurement and wiring reference.

\begin{table}[H]
  \centering
  \small
  \resizebox{\textwidth}{!}{%
    \begin{tabular}{@{}r l p{0.6\textwidth}@{}}
      \hline
      Qty & Component & Notes \\ \hline
      1 & Microcontroller & Handles sensors, actuators, and serial communication \\
      1 & Servo motor & Used as the hangar door actuator \\
      1 & Distance sensor (DDD) & Measures the drone's distance inside the hangar \\
      1 & Presence sensor (DPD) & Detects the drone approaching the hangar \\
      1 & Temperature sensor & Monitors internal hangar temperature \\
      3 & LEDs (L1, L2, L3) & System status indicators (normal, activity, alarm) \\
      3 & Resistors 220\,$\Omega$ & Current-limiting resistors for LEDs L1--L3 \\
      1 & LCD display & Displays system messages \\
      1 & RESET button & Clears alarms and restores normal operation \\
      1 & Resistor 10\,k$\Omega$ & Pull-up / pull-down resistor for the RESET button \\ \hline
    \end{tabular}%
  }
  \caption{Essential hardware list}
  \label{tab:hardware}
\end{table}

Figure~\ref{fig:hangar} shows the wiring diagram of the hangar control unit,
highlighting the main connections between the microcontroller, sensors,
actuators, and indicators.

\begin{figure}[H]
  \centering
  \includegraphics[width=0.8\textwidth]{./img/wiring.png}
  \caption{Wiring diagram of the hangar control unit.}
  \label{fig:hangar}
\end{figure}

\chapter{Architecture}

The system architecture is organized around a set of concurrent
tasks, each responsible for a specific function of the Smart Drone
Hangar. A lightweight cooperative scheduler periodically activates
these tasks in a cyclic order.

All tasks interact through a shared data structure called the
\textit{Context}. It acts as a shared memory, enabling coordination
between components via Boolean flags and message codes, ensuring
modularity and preventing direct coupling between tasks.

\section{System Task}
This task manages the global operating mode of the hangar, switching
between three states: \texttt{NORMAL}, \texttt{PREALARM}, and \texttt{ALARM}.

In \texttt{NORMAL}, all flight operations are allowed. If the
temperature exceeds \texttt{TEMP1} for longer than \texttt{T1}, the
system enters \texttt{PREALARM}, suspending new take-offs and
landings but allowing ongoing ones to complete. If the temperature
remains above \texttt{TEMP2} for more than \texttt{T2}, the system
transitions to \texttt{ALARM}: the hangar door is closed, the alarm
LED (L3) is activated, and all operations are halted. The system
remains in this state until the \textbf{RESET} button is pressed,
which returns it to \texttt{NORMAL}. If the drone is outside during
an alarm, an alert message is sent via the DRU.

\begin{figure}[H]
  \centering
  \includegraphics[width=\textwidth]{./pdf/SystemTask.pdf}
  \caption{System Task state diagram.}
  \label{fig:SystemTask}
\end{figure}

\section{Flight Task}
The \textbf{FlightTask} manages both take-off and landing operations, acting as the
intermediate layer between the system logic and the actuator control.  
It operates through three main states: \texttt{IDLE}, \texttt{WAITING}, and
\texttt{CHECKING}.

At system startup, the task initializes directly in the \texttt{CHECKING} state,
where it evaluates the sensor readings to determine whether a flight operation is in
progress or complete.  
This mechanism ensures that the system remains consistent with the physical state of
the hangar even after a power loss or reset.

During normal operation, when a command is received from the Drone Remote Unit (DRU),
the task transitions to \texttt{WAITING}.  
In this state, timers are initialized and the actuators are prepared.  
If valid conditions are detected within the timeout period (\(T_5\)) — for example, a
take-off message combined with a safe distance or a landing request detected by the
PIR sensor — the system moves to the \texttt{CHECKING} state.

In \texttt{CHECKING}, the FlightTask monitors the distance sensor to verify the
progress of the operation.  
A take-off is confirmed when the measured distance remains above the threshold
\(D_1\) for longer than \(T_3\), while landing is confirmed when the distance remains
below \(D_2\) for longer than \(T_4\).  
Upon completion, the gate closes automatically, LED~L2 stops blinking, and the system
returns to the \texttt{IDLE} state.

This centralized structure allows the FlightTask to coordinate both flight phases in
a single unified process, simplifying the control logic and maintaining full
synchronization between software behaviour and the physical state of the hangar.

\begin{figure}[H]
  \centering
  \includegraphics[width=\textwidth]{./pdf/FlightTask.pdf}
  \caption{Flight Task state diagram.}
  \label{fig:FlightTask}
\end{figure}

\newpage
\section{Blink Task}
The \textbf{BlinkTask} handles LED~L2 blinking, providing visual
feedback during active flight phases. It operates in two alternating
states, \texttt{ON} and \texttt{OFF}. At each activation, the task
toggles the LED state; however, a guard condition prevents the
transition from \texttt{OFF} to \texttt{ON} when blinking is
disabled, keeping the LED off during idle periods.

\begin{figure}[H]
  \centering
  \includegraphics[width=\textwidth]{./pdf/BlinkTask.pdf}
  \caption{Blink Task state diagram.}
  \label{fig:BlinkTask}
\end{figure}

\section{Gate Task}
The \textbf{GateTask} controls the hangar door using four states:
\texttt{CLOSE}, \texttt{OPENING}, \texttt{OPEN}, and
\texttt{CLOSING}. At startup, the gate is in \texttt{CLOSE}. When
enabled, it transitions to \texttt{OPENING} until the door is fully
open, then to \texttt{OPEN}. If the enable signal becomes false, it
moves to \texttt{CLOSING} until the door is fully closed. If a new
command arrives while the door is moving, the task immediately
reverses its direction.

\begin{figure}[H]
  \centering
  \includegraphics[width=\textwidth]{./pdf/GateTask.pdf}
  \caption{Gate Task state diagram.}
  \label{fig:GateTask}
\end{figure}

\newpage
\section{Observer Task}
The \textbf{ObserverTask} periodically monitors sensors and
communication channels by
evaluating a predicate function at each activation.
When the predicate evaluates to true, the associated callback is
executed, allowing the
system to react immediately to specific events such as sensor updates
or incoming
commands.
This provides a lightweight mechanism for asynchronous observation
without interfering
with the main control logic.

\begin{figure}[H]
  \centering
  \includegraphics[width=0.5\textwidth]{./pdf/ObserverTask.pdf}
  \caption{Observer Task state diagram.}
  \label{fig:ObserverTask}
\end{figure}

The Observer acts as a central listener: it periodically reads inputs
and messages
(e.g., from the DRU via serial communication) and dispatches relevant
information to the
appropriate tasks — System, Flight, Gate, etc.
This approach keeps each functional task lightweight and focused on
its own logic.
Moreover, the architecture supports multiple SystemTask instances,
each with its own
Context, enabling a single microcontroller to manage multiple hangars
in parallel and
simplifying message routing and scalability.

\chapter{Arduino Workflow}

\section{Overview}
This chapter describes the dynamic behaviour of the Smart Drone Hangar once deployed
on the Arduino platform.  
Each system function is implemented as an independent task, periodically activated by
a cooperative scheduler.  
Although tasks are executed sequentially, their high activation rate makes the system
respond as if all processes were concurrent, continuously reacting to inputs and
updating outputs in real time.

All components share a common data structure, the \textbf{Context}, which represents
the global state of the hangar.  
Through this structure, tasks communicate indirectly by setting and reading Boolean
flags that describe the hangar’s condition, the drone’s position, and the current
operating mode.

\section{System Dynamics}
At startup, the system initializes all hardware components and performs a sensor
check to determine the drone’s position.  
Depending on the readings from the distance (DDD) and presence (DPD) sensors, the
Context is initialized to either \texttt{DRONE INSIDE} or \texttt{DRONE OUTSIDE}.  
This ensures that the software state is immediately aligned with the physical
situation, even after a reset or power interruption.

The scheduler then maintains continuous control by cyclically activating all tasks to
supervise sensors, actuators, and communication interfaces.  
The \textbf{SystemTask} governs the overall behaviour of the hangar.  
It monitors temperature readings and selects the appropriate operating mode.  
Under normal conditions, flight operations are permitted.  
If the temperature exceeds the first threshold (\texttt{Temp1}) for longer than
\texttt{T1}, the system enters a pre-alarm state, suspending new take-offs or
landings while allowing any ongoing operation to finish.  
If the temperature rises above the higher threshold (\texttt{Temp2}) for more than
\texttt{T2}, the system switches to alarm mode: the gate closes, the alarm indicator
(LED~L3) turns on, and all operations remain halted until the \textbf{RESET} button is
pressed.  
This mechanism ensures that all flight activity takes place only under safe
environmental conditions.

\section{Coordination of Operations}
The \textbf{FlightTask} coordinates both take-off and landing operations, acting as
the intermediate layer between the system logic and the actuator control.  
It operates through three main states: \texttt{IDLE}, \texttt{WAITING}, and
\texttt{CHECKING}.  
At system startup, the task intentionally begins in the \texttt{CHECKING} state,
where it evaluates sensor readings to verify the current condition of the hangar and
restore consistency between the software and physical environment.

When a flight command is received from the Drone Remote Unit (DRU), the FlightTask
transitions to the \texttt{WAITING} state, initializing timers and preparing the
actuators.  
If valid conditions are detected within the timeout (\(T_5\))—for instance, a take-off
message combined with sufficient clearance or a landing detected by the PIR sensor—the
task moves to \texttt{CHECKING}.  
Here it monitors distance measurements to confirm the progress of the operation.  
A take-off is completed when the measured distance remains greater than \(D_1\) for
longer than \(T_3\), while a landing is confirmed when the distance stays below \(D_2\)
for at least \(T_4\).  
Once the operation ends, the gate closes, LED~L2 stops blinking, and the system
returns to the \texttt{IDLE} state.

This structure allows the FlightTask to manage both flight phases autonomously,
maintaining a consistent and fault-tolerant coordination between the drone’s motion
and the hangar’s actuators.

\section{Supporting Tasks and Context Interaction}
Several supporting tasks ensure system responsiveness and modularity.  
The \textbf{GateTask} drives the servo motor that opens or closes the hangar gate,
while the \textbf{BlinkTask} controls LEDs to indicate activity during flight
operations.  
The \textbf{ObserverTask} continuously reads sensor data and serial messages,
updating the Context with temperature, distance, and command information.

All tasks interact exclusively through the Context, which acts as shared memory and a
coordination channel.  
For example, the SystemTask may block new flights by setting a flag interpreted by the
FlightTask, while the ObserverTask records flight commands that will be processed in
the next scheduler cycle.

\section{Global Workflow}
The overall operation arises from the cooperation of all tasks under the scheduler’s
control.  
The ObserverTask updates the Context with real-time sensor data, the SystemTask
enforces safety conditions, and the FlightTask coordinates flight sequences by
commanding the Gate and Blink tasks according to the current state.  
When a sequence is completed, the system automatically returns to a safe standby mode,
ready for the next command.  
Through this cooperative and event-driven design, the Smart Drone Hangar achieves
robust, modular, and deterministic behaviour, maintaining continuous operation even
after temporary faults or power interruptions.

\appendix
\chapter{User Interface}

The user interface of the application is organized into three main panels,
each serving a distinct purpose within the system.

\section{Connection Panel}
The upper part of the view is dedicated to managing the connection
with the serial
port. The user can select the desired port, configure the
communication parameters,
and establish or terminate the connection with the device.

\section{Control Panel}
The central part of the interface provides simple controls for
operating the drone.
It includes two buttons — \textit{Takeoff} and \textit{Landing} —
which send the corresponding
commands to the controller. These buttons are automatically enabled or disabled
based on the current connection status and system state.

\section{Status Panel}
The lower part of the view displays real-time information about the
\textit{Hangar} and \textit{Drone} states, allowing continuous monitoring of the
overall system status.

\begin{figure}[H]
  \centering
  \includegraphics[width=\textwidth]{./img/app.png}
  \caption{App view}
  \label{fig:app}
\end{figure}

\chapter{User Guide}

\section{Cloning the repository}
Clone the project from GitHub and change to the project directory:
\begin{Verbatim}[tabsize=4,xleftmargin=2em]
> git clone https://github.com/alessandrorebosio/ESIOT25.git
> cd ESIOT25/assignment-02
\end{Verbatim}

\section{Connecting and starting the application}
Connect the Arduino to the computer and upload (flash) the firmware
to the board using the Arduino IDE or your preferred upload tool.
Identify the serial port used by the board and then start the application with:

\begin{Verbatim}[tabsize=4,xleftmargin=2em]
> java -jar drone-hangar-unit-all.jar
\end{Verbatim}

\vspace{\baselineskip}
\noindent Alternatively you can start the project with:
\begin{Verbatim}[tabsize=4,xleftmargin=2em]
> ./gradlew run
\end{Verbatim}

\end{document}
