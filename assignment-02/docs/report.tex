\documentclass[12pt,a4paper]{report}

\usepackage{alltt, fancyvrb, url}
\usepackage{graphicx}
\usepackage[utf8]{inputenc}
\usepackage{float}
\usepackage{xcolor}
\usepackage{hyperref}
\usepackage{longtable}
\usepackage{listings}
\usepackage{color}
\usepackage[utf8]{inputenc}

\usepackage{enumitem}
\usepackage{geometry}
\usepackage{pdfpages}

\geometry{margin=1in}

\definecolor{codegray}{rgb}{0.5,0.5,0.5}
\definecolor{codepurple}{rgb}{0.58,0,0.82}
\definecolor{backcolour}{rgb}{0.95,0.95,0.92}

\lstdefinestyle{sqlstyle}{
  language=SQL,
  backgroundcolor=\color{backcolour},
  commentstyle=\color{codegreen},
  keywordstyle=\color{codepurple},
  numberstyle=\numberstyle,
  stringstyle=\color{codepurple},
  basicstyle=\footnotesize\ttfamily,
  breakatwhitespace=false,
  breaklines=true,
  captionpos=b,
  keepspaces=true,
  numbers=left,
  numbersep=10pt,
  showspaces=false,
  showstringspaces=false,
  showtabs=false
}
\newcommand\numberstyle[1]{%
  \footnotesize
  \color{codegray}%
  \ttfamily
  \ifnum#1<10 0\fi#1 |%
}

\usepackage{textcomp}
\usepackage{siunitx}

\usepackage[english]{babel}
\usepackage[capitalise, english]{cleveref}

\graphicspath{ {./src/img} }

\textwidth=450pt\oddsidemargin=0pt
\begin{document}

\begin{titlepage}
  \begin{center}
    {{\Large{\textsc{Alma Mater Studiorum $\cdot$ University of
    Bologna}}}}\\[1ex]
    \rule{15.8cm}{0.6mm}\\[2ex]
    {\small Degree Programme in Engineering and Computer Science}\\
    {\small{ A.A. 2025/26}}
  \end{center}

  \vfill
  \begin{center}
    {\LARGE{\bf Smart Drone Hangar}}\\[1ex]
  \end{center}
  \vfill

  \begin{center}
    \begin{tabular}{@{}c@{\qquad}c@{}}
      {\large\bf Grazia Bochdanovits de Kavna} & {\large\bf
      Alessandro Rebosio} \\[0.5ex]
      {\small matr.\ 0001117082} & {\small matr.\ 0001130557}
    \end{tabular}
  \end{center}

  \vspace{2mm}
  \noindent\rule{\linewidth}{0.4pt}

\end{titlepage}

\tableofcontents

\chapter{Analysis}

\section{Description and requirements}
Initially the system starts with the hangar door HD closed; the DRONE
INSIDE state holds, L1 on, L2 and L3 off, and the LCD shows DRONE INSIDE.

\vspace{\baselineskip}
Take-off: the drone requests opening via DRU. On command the HD
opens, the LCD shows TAKE OFF, and the system waits for exit. Exit is
detected by the DDD: when distance {$>$} D1 for more than T1 the drone is
assumed out, the  HD closes and the LCD shows DRONE OUT.

\vspace{\baselineskip}
Landing: the drone requests opening via DRU. If DPD detects presence,
the HD opens and the LCD shows LANDING. When DDD measures distance {$ < $}
D2 for more than T2 the drone is landed, the HD closes and the LCD
shows DRONE INSIDE.

\vspace{\baselineskip}
\noindent During take-off/landing L2 blinks (0.5 s period); otherwise it is off.

\vspace{\baselineskip}
Temperature monitoring runs whenever the drone is inside (rest,
take-off, landing). If temperature $\geq$ Temp1 for more than T3 the
system enters pre-alarm: new take-offs/landings are suspended until
return to normal operation (in-progress operations may complete). If
temperature $\geq$ Temp2 ($>$ Temp1) for more than T4 the HD is
closed (if open), L3 turns on and the LCD shows ALARM. If the drone
is outside, an ALARM message is sent via DRU. All operations stay
suspended until the RESET button is pressed; pressing it returns the
system to normal operation.

\vspace{\baselineskip}
\noindent Parameters D1, D2, T1, T2, T3, T4, Temp1, Temp2 are left
configurable for testing.

The DRU GUI must allow:
\begin{itemize}
  \item sending take-off/landing commands (simulate the drone);
  \item displaying drone state (rest, taking off, operating, landing);
  \item displaying hangar state (normal, ALARM);
  \item (during landing) showing current distance to ground.
\end{itemize}

\section{Wiring}
Below is a concise list of the essential hardware components required
to build and test the Smart Drone Hangar. Quantities, components and
brief notes on their function are provided to assist with procurement
and wiring.

\begin{table}[H]
  \centering
  \small
  \resizebox{\textwidth}{!}{%
    \begin{tabular}{@{}r l p{0.6\textwidth}@{}}
      \hline
      Qty & Component & Notes \\
      \hline
      1 & Microcontroller & Handles sensors, actuators, and serial
      communication \\
      1 & Servo motor & Used as the hangar door actuator \\
      1 & Distance sensor (DDD) & Measures the drone's distance
      inside the hangar \\
      1 & Presence sensor (DPD) & Detects the drone approaching the hangar \\
      1 & Temperature sensor & Monitors internal hangar temperature \\
      3 & LEDs (L1, L2, L3) & System status indicators (normal,
      activity, alarm) \\
      3 & Resistors 220\,$\Omega$ & Current-limiting resistors for LEDs
      L1, L2, L3 \\
      1 & LCD display & Displays system messages \\
      1 & RESET button & Used to clear alarms and restore normal operation \\
      1 & Resistor 10\,k$\Omega$ & Pull-up / pull-down resistor for the
      RESET button \\
      \hline
    \end{tabular}%
  }
  \caption{Essential hardware list}
\end{table}

Figure \ref{fig:hangar} shows the wiring diagram of the hangar
control unit, highlighting the main connections between the
microcontroller, sensors, actuators, and indicators.
\begin{figure}[H]
  \centering
  \includegraphics[width=\textwidth]{./img/wiring.png}
  \caption{Wiring diagram of the hangar control unit.}
  \label{fig:hangar}
\end{figure}

\chapter{Architecture}
The implementation is organized into multiple tasks, each responsible
for a specific function of the project. A simple scheduler was
implemented to run these tasks concurrently. The main tasks are described below.

\section{System Task}
The system operates in three main states: \texttt{NORMAL}, \texttt{PREALARM}, and \texttt{ALARM}, with transitions driven by the internal temperature and timing conditions. 
In the \texttt{NORMAL} state, the system enables take-off and landing operations. 
If the temperature exceeds \texttt{TEMP1} for a continuous period longer than \texttt{T1}, the system transitions to the \texttt{PREALARM} state; otherwise, the timer is reset whenever the temperature drops below \texttt{TEMP1}. 

In the \texttt{PREALARM} state, the system suspends new flight operations while allowing in-progress ones to complete. 
If the temperature rises above \texttt{TEMP2} for more than \texttt{T2}, the system enters the \texttt{ALARM} state; if it falls below \texttt{TEMP1}, the timer is reset and the system returns to \texttt{NORMAL}. 

In the \texttt{ALARM} state, the system closes the hangar door if it is open. 
If the drone is outside when the alarm is triggered, an alarm message is sent via \texttt{DRU}. 
The system remains in the \texttt{ALARM} state until the \texttt{RESET} button is pressed, which returns the system to the \texttt{NORMAL} state. 

\begin{figure}[H]
  \centering
  \includegraphics[width=\textwidth]{./pdf/SystemTask.pdf}
  \caption{System Task state diagram.}
  \label{fig:SystemTask}
\end{figure}
Note: when the entry action \texttt{/AllowFlight} is
executed, a boolean flag is set to \texttt{true}; when the exit
action \texttt{/BlockFlight} is executed, the flag is set to \texttt{false}.

\section{Flight Task}

The \texttt{FlightTask} manages the drone behavior three states: 
\texttt{IDLE}, \texttt{WAITING}, and \texttt{OPERATING}.
The task runs only when the flight system is enabled.  
During initialization, the starting state is chosen dynamically: if the operation
is already completed (\texttt{isOperationDone()} returns true), the task starts
in \texttt{IDLE}; otherwise, it starts in \texttt{OPERATING}.

In \texttt{IDLE}, the system waits for commands. When a take-off message is received
and the measured distance is below \texttt{D1}, the task switches to
\texttt{OPERATING}, starting the take-off sequence. If a landing message is received,
it transitions to \texttt{WAITING}.

In \texttt{WAITING}, the task waits up to \texttt{T5} milliseconds for the drone to be
detected. If detection occurs, it performs the landing actions and moves to
\texttt{OPERATING}; otherwise, after the timeout, it returns to \texttt{IDLE}.

In \texttt{OPERATING}, the gate is opened and a blinking signal is activated.
When the operation is completed, the task stops the blinking, closes the gate, and returns to \texttt{IDLE}.

\begin{figure}[H]
  \centering
  \includegraphics[width=\textwidth]{./pdf/FlightTask.pdf}
  \caption{Flight Task state diagram.}
  \label{fig:FlightTask}
\end{figure}
Note: the actions \texttt{/close} and \texttt{/stopBlink} stop the Gate
and Blink tasks respectively; the actions \texttt{/startBlink} and
\texttt{/open} start those tasks.

\section{Check Task}
The Check Task is responsible for monitoring and maintaining the
drone state. At startup it reads the presence/distance sensors and
sets the initial state to \texttt{DRONE\_INSIDE} if the drone is
detected, or to \texttt{DRONE\_OUTSIDE} otherwise.

\begin{figure}[H]
  \centering
  \includegraphics[width=\textwidth]{./pdf/CheckTask.pdf}
  \caption{Check Task state diagram.}
  \label{fig:CheckTask}
\end{figure}

\section{Blink Task}
It implements two states: \texttt{ON} and \texttt{OFF}. On each
invocation the task toggles its state and updates the LED output
accordingly. A guard condition prevents the transition from
\texttt{OFF} to \texttt{ON} when blinking is disabled, so the LED
remains off when blinking is not allowed.

\begin{figure}[H]
  \centering
  \includegraphics[width=\textwidth]{./pdf/BlinkTask.pdf}
  \caption{Blinking Task state diagram.}
  \label{fig:BlinkingTask}
\end{figure}

\section{Gate Task}
The \texttt{Gate Task} manages the hangar gate through four operational states:
\texttt{CLOSE}, \texttt{OPENING}, \texttt{OPEN}, and \texttt{CLOSING}. 
At system startup, the gate is in the \texttt{CLOSE} state, where the motor is turned off. 
Upon receiving an \texttt{enabled} signal, the task transitions to the \texttt{OPENING} state, 
where the gate starts to open. Once the opening process is completed, 
the system moves to the \texttt{OPEN} state, in which the motor is turned off and the gate remains open. 

If the \texttt{enabled} condition becomes false while the gate is open, the task transitions to the \texttt{CLOSING} state, 
closing the gate until the \texttt{CLOSE} state is reached. 

If an opposite command is received while the gate is in motion, 
the direction is immediately reversed, allowing for dynamic control of gate movement.

\begin{figure}[H]
  \centering
  \includegraphics[width=\textwidth]{./pdf/GateTask.pdf}
  \caption{Gate Task state diagram.}
  \label{fig:GateTask}
\end{figure}

\section{Observer Task}
The \texttt{Observer Task} is responsible for periodically monitoring system conditions through the evaluation of a predicate function. 
It operates in a single state, \texttt{OBSERVER}, where at each activation the associated predicate is checked. 
If the predicate evaluates to true, the corresponding function or callback is executed, allowing the system to react to specific events or conditions in real time. 
This task provides a lightweight mechanism for asynchronous observation and event detection without interfering with the main control logic.

\begin{figure}[H]
  \centering
  \includegraphics[width=0.6\textwidth]{./pdf/ObserverTask.pdf}
  \caption{Observer Task state diagram.}
  \label{fig:ObserverTask}
\end{figure}

\chapter{Diagrams}

\section{Application class}

\section{Arduino module}

\appendix
\chapter{User Interface}

The user interface of the application is organized into three main panels, 
each serving a distinct purpose within the system.

\section{Connection Panel}
The upper part of the view is dedicated to managing the connection with the serial 
port. The user can select the desired port, configure the communication parameters, 
and establish or terminate the connection with the device.

\section{Control Panel}
The central part of the interface provides simple controls for operating the drone.
It includes two buttons — \textit{Takeoff} and \textit{Landing} — which send the corresponding 
commands to the controller. These buttons are automatically enabled or disabled 
based on the current connection status and system state.

\section{Status Panel}
The lower part of the view displays real-time information about the 
\textit{Hangar} and \textit{Drone} states, allowing continuous monitoring of the 
overall system status.


\chapter{User Guide}

\section{Cloning the repository}
Clone the project from GitHub and change to the project directory:
\begin{Verbatim}[tabsize=4,xleftmargin=2em]
> git clone https://github.com/alessandrorebosio/ESIOT25.git
> cd ESIOT25/assignment-02
\end{Verbatim}

\section{Connecting and starting the application}
Connect the Arduino to your computer, identify the serial port, then
start the application with:
\begin{Verbatim}[tabsize=4,xleftmargin=2em]
> java -jar drone-hangar-unit-all.jar
\end{Verbatim}

\vspace{\baselineskip}
\noindent Alternatively you can start the project with:
\begin{Verbatim}[tabsize=4,xleftmargin=2em]
> ./gradlew run
\end{Verbatim}

\end{document}
