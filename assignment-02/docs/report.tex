\documentclass[12pt,a4paper]{report}

\usepackage{alltt, fancyvrb, url}
\usepackage{graphicx}
\usepackage[utf8]{inputenc}
\usepackage{float}
\usepackage{xcolor}
\usepackage{hyperref}
\usepackage{longtable}
\usepackage{listings}
\usepackage{color}
\usepackage[utf8]{inputenc}

\usepackage{enumitem}
\usepackage{geometry}
\usepackage{pdfpages}

\geometry{margin=1in}

\definecolor{codegray}{rgb}{0.5,0.5,0.5}
\definecolor{codepurple}{rgb}{0.58,0,0.82}
\definecolor{backcolour}{rgb}{0.95,0.95,0.92}

\lstdefinestyle{sqlstyle}{
  language=SQL,
  backgroundcolor=\color{backcolour},
  commentstyle=\color{codegreen},
  keywordstyle=\color{codepurple},
  numberstyle=\numberstyle,
  stringstyle=\color{codepurple},
  basicstyle=\footnotesize\ttfamily,
  breakatwhitespace=false,
  breaklines=true,
  captionpos=b,
  keepspaces=true,
  numbers=left,
  numbersep=10pt,
  showspaces=false,
  showstringspaces=false,
  showtabs=false
}
\newcommand\numberstyle[1]{%
  \footnotesize
  \color{codegray}%
  \ttfamily
  \ifnum#1<10 0\fi#1 |%
}

\usepackage{textcomp}
\usepackage{siunitx}

\usepackage[english]{babel}
\usepackage[capitalise, english]{cleveref}

\graphicspath{ {./src/img} }

\textwidth=450pt\oddsidemargin=0pt
\begin{document}

\begin{titlepage}
  \begin{center}
    {{\Large{\textsc{Alma Mater Studiorum $\cdot$ University of
    Bologna}}}}\\[1ex]
    \rule{15.8cm}{0.6mm}\\[2ex]
    {\small Degree Programme in Engineering and Computer Science}\\
    {\small{ A.A. 2025/26}}
  \end{center}

  \vfill
  \begin{center}
    {\LARGE{\bf Smart Drone Hangar}}\\[1ex]
  \end{center}
  \vfill

  \begin{center}
    \begin{tabular}{@{}c@{\qquad}c@{}}
      {\large\bf Grazia Bochdanovits de Kavna} & {\large\bf
      Alessandro Rebosio} \\[0.5ex]
      {\small matr.\ 0001117082} & {\small matr.\ 0001130557}
    \end{tabular}
  \end{center}

  \vspace{2mm}
  \noindent\rule{\linewidth}{0.4pt}

\end{titlepage}

\tableofcontents

\chapter{Analysis}

\section{Description and requirements}
Initially the system starts with the hangar door HD closed; the DRONE
INSIDE state holds, L1 on, L2 and L3 off, and the LCD shows DRONE INSIDE.

\vspace{\baselineskip}
Take-off: the drone requests opening via DRU. On command the HD
opens, the LCD shows TAKE OFF, and the system waits for exit. Exit is
detected by the DDD: when distance {$>$} D1 for more than T1 the drone is
assumed out, the  HD closes and the LCD shows DRONE OUT.

\vspace{\baselineskip}
Landing: the drone requests opening via DRU. If DPD detects presence,
the HD opens and the LCD shows LANDING. When DDD measures distance {$ < $}
D2 for more than T2 the drone is landed, the HD closes and the LCD
shows DRONE INSIDE.

\vspace{\baselineskip}
\noindent During take-off/landing L2 blinks (0.5 s period); otherwise it is off.

\vspace{\baselineskip}
Temperature monitoring runs whenever the drone is inside (rest,
take-off, landing). If temperature $\geq$ Temp1 for more than T3 the
system enters pre-alarm: new take-offs/landings are suspended until
return to normal operation (in-progress operations may complete). If
temperature $\geq$ Temp2 ($>$ Temp1) for more than T4 the HD is
closed (if open), L3 turns on and the LCD shows ALARM. If the drone
is outside, an ALARM message is sent via DRU. All operations stay
suspended until the RESET button is pressed; pressing it returns the
system to normal operation.

\vspace{\baselineskip}
\noindent Parameters D1, D2, T1, T2, T3, T4, Temp1, Temp2 are left
configurable for testing.

The DRU GUI must allow:
\begin{itemize}
  \item sending take-off/landing commands (simulate the drone);
  \item displaying drone state (rest, taking off, operating, landing);
  \item displaying hangar state (normal, ALARM);
  \item (during landing) showing current distance to ground.
\end{itemize}

\section{Wiring}
Below is a concise list of the essential hardware components required
to build and test the Smart Drone Hangar. Quantities, components and
brief notes on their function are provided to assist with procurement
and wiring.

\begin{table}[H]
  \centering
  \small
  \resizebox{\textwidth}{!}{%
    \begin{tabular}{@{}r l p{0.6\textwidth}@{}}
      \hline
      Qty & Component & Notes \\
      \hline
      1 & Microcontroller & Handles sensors, actuators, and serial
      communication \\
      1 & Servo motor & Used as the hangar door actuator \\
      1 & Distance sensor (DDD) & Measures the drone's distance
      inside the hangar \\
      1 & Presence sensor (DPD) & Detects the drone approaching the hangar \\
      1 & Temperature sensor & Monitors internal hangar temperature \\
      3 & LEDs (L1, L2, L3) & System status indicators (normal,
      activity, alarm) \\
      3 & Resistors 220\,$\Omega$ & Current-limiting resistors for LEDs
      L1, L2, L3 \\
      1 & LCD display & Displays system messages \\
      1 & RESET button & Used to clear alarms and restore normal operation \\
      1 & Resistor 10\,k$\Omega$ & Pull-up / pull-down resistor for the
      RESET button \\
      \hline
    \end{tabular}%
  }
  \caption{Essential hardware list}
\end{table}

Figure \ref{fig:hangar} shows the wiring diagram of the hangar
control unit, highlighting the main connections between the
microcontroller, sensors, actuators, and indicators.
\begin{figure}[H]
  \centering
  \includegraphics[width=\textwidth]{./img/wiring.png}
  \caption{Wiring diagram of the hangar control unit.}
  \label{fig:hangar}
\end{figure}

\chapter{Architecture}
The implementation is organized into multiple tasks, each responsible
for a specific function of the project. A simple scheduler was
implemented to run these tasks concurrently. The main tasks are described below.

\section{System Task}
The system defines three main states: \texttt{NORMAL},
\texttt{PREALARM}, and \texttt{ALARM}. In the \texttt{NORMAL} state
take-offs and landings are allowed. If the internal temperature
remains greater than or equal to \texttt{Temp1} for more than
\texttt{T3}, the system transitions to \texttt{PREALARM}, where new
take-off and landing requests are suspended while in-progress
operations may complete. If the temperature then remains greater than
or equal to \texttt{Temp2} (with \texttt{Temp2} $>$ \texttt{Temp1}) for
more than \texttt{T4}, the system enters the \texttt{ALARM} state:
the hangar door is closed if open, \texttt{L3} is turned on, and the
LCD displays \texttt{ALARM}. If the drone is outside when the alarm
triggers, an alarm message is sent via \texttt{DRU}. All operations
remain suspended in \texttt{ALARM} until the \texttt{RESET} button is
pressed, which returns the system to \texttt{NORMAL}. 

\begin{figure}[H]
  \centering
  \includegraphics[width=\textwidth]{./pdf/SystemTask.pdf}
  \caption{System Task state diagram.}
  \label{fig:SystemTask}
\end{figure}

\section{Flight Task}

\begin{figure}[H]
  \centering
  \includegraphics[width=\textwidth]{./pdf/FlightTask.pdf}
  \caption{Flight Task state diagram.}
  \label{fig:FlightTask}
\end{figure}

\section{Check Task}

\begin{figure}[H]
  \centering
  \includegraphics[width=\textwidth]{./pdf/CheckTask.pdf}
  \caption{Check Task state diagram.}
  \label{fig:CheckTask}
\end{figure}

\section{Blink Task}
The blinking task controls the LED used for activity indication. It
implements two states, On and Off. On each invocation the task
toggles its state and updates the LED output accordingly.

\begin{figure}[H]
  \centering
  \includegraphics[width=\textwidth]{./pdf/BlinkTask.pdf}
  \caption{Blinking Task state diagram.}
  \label{fig:BlinkingTask}
\end{figure}

\section{Gate Task}

\begin{figure}[H]
  \centering
  \includegraphics[width=\textwidth]{./pdf/GateTask.pdf}
  \caption{Gate Task state diagram.}
  \label{fig:GateTask}
\end{figure}

\chapter{Diagrams}

\section{Application class}

\section{Arduino module}

\appendix
\chapter{User Guide}

\section{Cloning the repository}
Clone the project from GitHub and change to the project directory:
\begin{Verbatim}[tabsize=4,xleftmargin=2em]
> git clone https://github.com/alessandrorebosio/ESIOT25.git
> cd ESIOT25/assignment-02
\end{Verbatim}

\section{Connecting and starting the application}
Connect the Arduino to your computer, identify the serial port, then
start the application with:
\begin{Verbatim}[tabsize=4,xleftmargin=2em]
> java -jar drone-hangar-unit-all.jar
\end{Verbatim}

\vspace{\baselineskip}
\noindent Alternatively you can start the project with:
\begin{Verbatim}[tabsize=4,xleftmargin=2em]
> ./gradlew run
\end{Verbatim}

\end{document}
