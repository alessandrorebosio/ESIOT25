\documentclass[12pt,a4paper]{report}

\usepackage{alltt, fancyvrb, url}
\usepackage{graphicx}
\usepackage[utf8]{inputenc}
\usepackage{float}
\usepackage{xcolor}
\usepackage{hyperref}
\usepackage{longtable}
\usepackage{listings}
\usepackage{color}
\usepackage[utf8]{inputenc}

\usepackage{enumitem}
\usepackage{geometry}
\usepackage{pdfpages}

\geometry{margin=1in}

\definecolor{codegray}{rgb}{0.5,0.5,0.5}
\definecolor{codepurple}{rgb}{0.58,0,0.82}
\definecolor{backcolour}{rgb}{0.95,0.95,0.92}

\lstdefinestyle{sqlstyle}{
  language=SQL,
  backgroundcolor=\color{backcolour},
  commentstyle=\color{codegreen},
  keywordstyle=\color{codepurple},
  numberstyle=\numberstyle,
  stringstyle=\color{codepurple},
  basicstyle=\footnotesize\ttfamily,
  breakatwhitespace=false,
  breaklines=true,
  captionpos=b,
  keepspaces=true,
  numbers=left,
  numbersep=10pt,
  showspaces=false,
  showstringspaces=false,
  showtabs=false
}
\newcommand\numberstyle[1]{%
  \footnotesize
  \color{codegray}%
  \ttfamily
  \ifnum#1<10 0\fi#1 |%
}

\usepackage{textcomp}
\usepackage{siunitx}

\usepackage[english]{babel}
\usepackage[capitalise, english]{cleveref}

\graphicspath{ {./src/img} }

\textwidth=450pt\oddsidemargin=0pt
\begin{document}

\begin{titlepage}
  \begin{center}
    {{\Large{\textsc{Alma Mater Studiorum $\cdot$ University of
    Bologna}}}}\\[1ex]
    \rule{15.8cm}{0.6mm}\\[2ex]
    {\small Degree Programme in Engineering and Computer Science}\\
    {\small{ A.A. 2025/26}}
  \end{center}

  \vfill
  \begin{center}
    {\LARGE{\bf Smart Drone Hangar}}\\[1ex]
  \end{center}
  \vfill

  \begin{center}
    \begin{tabular}{@{}c@{\qquad}c@{}}
      {\large\bf Grazia Bochdanovits de Kavna} & {\large\bf
      Alessandro Rebosio} \\[0.5ex]
      {\small matr.\ 0001117082} & {\small matr.\ 0001130557}
    \end{tabular}
  \end{center}

  \vspace{2mm}
  \noindent\rule{\linewidth}{0.4pt}

\end{titlepage}

\tableofcontents

\chapter{Analysis}

\section{Description and Requirements}

Initially, the hangar door (HD) is closed and the system is in the
\texttt{DRONE INSIDE} state.  
LED~L1 is on, LEDs~L2 and~L3 are off, and the LCD displays
``\texttt{DRONE INSIDE}''.

\vspace{\baselineskip}
\noindent\textbf{Take-off Sequence.}  
When the drone requests take-off through the Drone Remote Unit (DRU), the hangar door
opens and the LCD displays ``\texttt{TAKE OFF}''.  
The system then waits for the drone to exit the hangar.  
Exit is detected by the distance sensor (DDD): when the measured distance exceeds
$D_1$ for more than $T_1$, the drone is considered outside.  
At that moment, the door closes automatically and the LCD displays
``\texttt{DRONE OUT}''.

\vspace{\baselineskip}
\noindent\textbf{Landing Sequence.}  
When the drone requests landing via the DRU, the presence sensor (DPD) detects its
approach.  
The hangar door opens and the LCD displays ``\texttt{LANDING}''.  
The landing is confirmed when the DDD measures a distance below $D_2$ for longer than
$T_2$.  
Once landed, the door closes and the LCD returns to
``\texttt{DRONE INSIDE}''.

\vspace{\baselineskip}
\noindent\textbf{Indicators and Lighting.}  
During take-off and landing operations, LED~L2 blinks with a 0.5\,s period to indicate
activity.  
In all other states it remains off.

\vspace{\baselineskip}
\noindent\textbf{Temperature Monitoring and Safety Conditions.}  
Temperature control is active whenever the drone is inside the hangar, regardless of the
current operation.  
If the internal temperature remains above \texttt{Temp1} for more than \texttt{T3}, the
system enters a \textbf{pre-alarm} condition: new take-offs or landings are suspended,
while any operation already in progress may complete.  
If the temperature rises above \texttt{Temp2} (with \texttt{Temp2 > Temp1}) for longer
than \texttt{T4}, the system switches to \textbf{alarm} mode.  
In this state, the hangar door is closed (if open), LED~L3 turns on, and the LCD shows
``\texttt{ALARM}''.  
If the drone is outside during an alarm, an ``\texttt{ALARM}'' message is sent via the
DRU.  
All operations remain suspended until the \textbf{RESET} button is pressed, which
restores the system to normal operation.

\vspace{\baselineskip}
\noindent\textbf{Configurable Parameters and GUI Requirements.}  
The parameters $D_1$, $D_2$, $T_1$, $T_2$, $T_3$, $T_4$, \texttt{Temp1}, and
\texttt{Temp2} are configurable for testing and calibration.  

The DRU graphical interface must provide the following functions:
\begin{itemize}
  \item Sending take-off and landing commands to simulate drone actions;
  \item Displaying the drone state (\emph{rest}, \emph{taking off}, \emph{operating},
        \emph{landing});
  \item Displaying the hangar state (\emph{normal} or \emph{alarm});
  \item Showing the current distance from the ground during landing.
\end{itemize}


\section{Wiring}
Below is a concise list of the essential hardware components required
to build and test the Smart Drone Hangar. Quantities, components and
brief notes on their function are provided to assist with procurement
and wiring.

\begin{table}[H]
  \centering
  \small
  \resizebox{\textwidth}{!}{%
    \begin{tabular}{@{}r l p{0.6\textwidth}@{}}
      \hline
      Qty & Component & Notes \\
      \hline
      1 & Microcontroller & Handles sensors, actuators, and serial
      communication \\
      1 & Servo motor & Used as the hangar door actuator \\
      1 & Distance sensor (DDD) & Measures the drone's distance
      inside the hangar \\
      1 & Presence sensor (DPD) & Detects the drone approaching the hangar \\
      1 & Temperature sensor & Monitors internal hangar temperature \\
      3 & LEDs (L1, L2, L3) & System status indicators (normal,
      activity, alarm) \\
      3 & Resistors 220\,$\Omega$ & Current-limiting resistors for LEDs
      L1, L2, L3 \\
      1 & LCD display & Displays system messages \\
      1 & RESET button & Used to clear alarms and restore normal operation \\
      1 & Resistor 10\,k$\Omega$ & Pull-up / pull-down resistor for the
      RESET button \\
      \hline
    \end{tabular}%
  }
  \caption{Essential hardware list}
\end{table}

Figure \ref{fig:hangar} shows the wiring diagram of the hangar
control unit, highlighting the main connections between the
microcontroller, sensors, actuators, and indicators.
\begin{figure}[H]
  \centering
  \includegraphics[width=0.8\textwidth]{./img/wiring.png}
  \caption{Wiring diagram of the hangar control unit.}
  \label{fig:hangar}
\end{figure}

\chapter{Architecture}

The system architecture is organized around a set of concurrent tasks, each responsible
for a specific function of the Smart Drone Hangar.  
A lightweight cooperative scheduler periodically activates these tasks in a cyclic order.
Although executed sequentially, their high activation rate allows the system to behave
as if all processes were running concurrently.

All tasks interact through a shared data structure called the \textit{Context}, which
maintains the global state of the hangar — including drone position, door status,
temperature conditions, and system flags.  
The \textit{Context} serves as a shared memory that enables coordination among
components via Boolean variables and message codes, ensuring modularity and preventing
direct coupling between tasks.

\vspace{\baselineskip}
The main functional components are:
\begin{itemize}
  \item \textbf{System Task} – manages the overall operating mode (\texttt{NORMAL},
        \texttt{PREALARM}, \texttt{ALARM});
  \item \textbf{Flight Task} – coordinates flight operations and activates the other
        control tasks;
  \item \textbf{Takeoff Task} and \textbf{Landing Task} – specialized observer tasks
        managing their respective flight phases;
  \item \textbf{Gate Task} – controls the hangar door actuator;
  \item \textbf{Blink Task} – manages LED blinking during active operations;
  \item \textbf{Observer Task} – monitors sensors and serial messages, updating the
        \textit{Context}.
\end{itemize}

\section{System Task}
The \textbf{SystemTask} manages the global operating mode of the hangar, switching
between three states: \texttt{NORMAL}, \texttt{PREALARM}, and \texttt{ALARM}.  

In \texttt{NORMAL}, all flight operations are allowed.  
If the temperature exceeds \texttt{TEMP1} for longer than \texttt{T1}, the system
enters \texttt{PREALARM}, suspending new take-offs and landings but allowing ongoing
ones to complete.  
If the temperature remains above \texttt{TEMP2} for more than \texttt{T2}, the system
transitions to \texttt{ALARM}: the hangar door is closed, the alarm LED (L3) is
activated, and all operations are halted.  
The system remains in this state until the \textbf{RESET} button is pressed, which
returns it to \texttt{NORMAL}.  
If the drone is outside during an alarm, an alert message is sent via the DRU.

\begin{figure}[H]
  \centering
  \includegraphics[width=\textwidth]{./pdf/SystemTask.pdf}
  \caption{System Task state diagram.}
  \label{fig:SystemTask}
\end{figure}

\vspace{\baselineskip}
When the entry action \texttt{/AllowFlight} is executed, a Boolean flag is set to
\texttt{true}; conversely, the exit action \texttt{/BlockFlight} sets it to
\texttt{false}.  
This flag informs other tasks whether flight operations are currently permitted.

\vspace{\baselineskip}
\section{Flight Task}
The \textbf{FlightTask} coordinates the take-off and landing sequences, acting as the
intermediate layer between the high-level system logic and the low-level actuator
control.  
It operates through three main states: \texttt{IDLE}, \texttt{WAITING}, and
\texttt{OPERATING}.

At system startup, the FlightTask determines its initial state based on Context flags:
if all operations are complete, it starts in \texttt{IDLE}; otherwise, it initializes
directly in \texttt{OPERATING}.  
This mechanism allows the hangar to resume ongoing operations after a power loss or
reset, ensuring consistency between software logic and physical state.

During normal operation, when a take-off or landing command is received, the task
activates the corresponding sequence by updating the Context.  
The hangar door opens and LED~L2 begins blinking.  
Once sensors confirm completion — e.g., the drone has exited or landed — the door
closes, blinking stops, and the system returns to \texttt{IDLE}.

\begin{figure}[H]
  \centering
  \includegraphics[width=\textwidth]{./pdf/FlightTask.pdf}
  \caption{Flight Task state diagram.}
  \label{fig:FlightTask}
\end{figure}

\vspace{\baselineskip}
When entering the \texttt{OPERATING} state, the FlightTask triggers the
\texttt{/startBlink} and \texttt{/openGate} actions, enabling the corresponding tasks.
Upon completion of the operation, it executes \texttt{/stopBlink} and
\texttt{/closeGate}, restoring the system to its resting configuration.

\vspace{\baselineskip}
\section{Takeoff Task}
The \textbf{TakeoffTask} is an observer responsible for verifying when the drone has
fully left the hangar.  
It monitors the distance sensor (DDD) and, when the measured distance exceeds
$D_1$ for longer than $T_1$, it marks the operation as complete in the Context.
This notification allows the FlightTask to terminate the sequence and return to the
\texttt{IDLE} state.

\vspace{\baselineskip}
\section{Landing Task}
The \textbf{LandingTask} monitors the approach and landing phase of the drone.  
Once the presence sensor (DPD) detects the drone and the distance measured by the DDD
remains below $D_2$ for more than $T_2$, the task signals successful landing by updating
the Context.  
The FlightTask then closes the hangar door and resumes standby mode.

\vspace{\baselineskip}
\section{Gate Task}
The \textbf{GateTask} controls the hangar door using four states:
\texttt{CLOSE}, \texttt{OPENING}, \texttt{OPEN}, and \texttt{CLOSING}.  
At startup, the gate is in \texttt{CLOSE}.  
When enabled, it transitions to \texttt{OPENING} until the door is fully open, then to
\texttt{OPEN}.  
If the enable signal becomes false, it moves to \texttt{CLOSING} until the door is fully
closed.  
If a new command arrives while the door is moving, the task immediately reverses its
direction.

\begin{figure}[H]
  \centering
  \includegraphics[width=\textwidth]{./pdf/GateTask.pdf}
  \caption{Gate Task state diagram.}
  \label{fig:GateTask}
\end{figure}

% \section{Check Task} 
% The Check Task is responsible for monitoring and maintaining the 
% drone state. At startup it reads the presence/distance sensors and 
% sets the initial state to \texttt{DRONE\_INSIDE} if the drone is 
% detected, or to \texttt{DRONE\_OUTSIDE} otherwise. 
% \begin{figure}[H] 
% \centering 
% \includegraphics[width=\textwidth]{./pdf/CheckTask.pdf} 
% \caption{Check Task state diagram.} 
% \label{fig:CheckTask} 
% \end{figure}

\vspace{\baselineskip}
\section{Blink Task}
The \textbf{BlinkTask} handles LED~L2 blinking, providing visual feedback during active
flight phases.  
It operates in two alternating states, \texttt{ON} and \texttt{OFF}.  
At each activation, the task toggles the LED state; however, a guard condition prevents
the transition from \texttt{OFF} to \texttt{ON} when blinking is disabled, keeping the
LED off during idle periods.

\begin{figure}[H]
  \centering
  \includegraphics[width=\textwidth]{./pdf/BlinkTask.pdf}
  \caption{Blink Task state diagram.}
  \label{fig:BlinkTask}
\end{figure}

\vspace{\baselineskip}
\section{Observer Task}
The \textbf{ObserverTask} periodically monitors sensors and communication channels by
evaluating a predicate function at each activation.  
When the predicate evaluates to true, the associated callback is executed, allowing the
system to react immediately to specific events such as sensor updates or incoming
commands.  
This provides a lightweight mechanism for asynchronous observation without interfering
with the main control logic.

\begin{figure}[H]
  \centering
  \includegraphics[width=0.8\textwidth]{./pdf/ObserverTask.pdf}
  \caption{Observer Task state diagram.}
  \label{fig:ObserverTask}
\end{figure}

\vspace{\baselineskip}
The Observer acts as a central listener: it periodically reads inputs and messages
(e.g., from the DRU via serial communication) and dispatches relevant information to the
appropriate tasks — System, Flight, Gate, etc.  
This approach keeps each functional task lightweight and focused on its own logic.  
Moreover, the architecture supports multiple SystemTask instances, each with its own
Context, enabling a single microcontroller to manage multiple hangars in parallel and
simplifying message routing and scalability.

\chapter{Arduino Workflow}

\section{Overview}
This chapter describes the dynamic behaviour of the Smart Drone Hangar once deployed
on the Arduino platform.  
Each function of the system is implemented as an independent task, periodically
activated by a simple cooperative scheduler.  
Although tasks are executed sequentially, their rapid activation rate makes the system
behave as if all processes were running concurrently, continuously reacting to inputs
and updating outputs in real time.

All components share a common data structure, the \textbf{Context}, which represents
the global state of the hangar.  
Through this structure, tasks communicate indirectly by setting and reading Boolean
flags that describe the hangar’s condition, the drone’s position, and the current
operating mode.

\vspace{\baselineskip}
\section{System Dynamics}
At startup, the system initializes all hardware components, ensuring a safe initial
state: the hangar door is closed, the drone is detected inside, and flight operations
are temporarily disabled until normal conditions are confirmed.  
From this point on, the scheduler maintains continuous control by activating all tasks
in sequence to supervise sensors, actuators, and communication interfaces.

The \textbf{SystemTask} governs the global behaviour of the hangar.  
It continuously monitors temperature readings and, according to the measured values,
selects the appropriate operating mode.  
During normal conditions, flights are allowed and all functions operate regularly.  
If the temperature exceeds the first threshold (\texttt{Temp1}) for longer than the
configured time (\texttt{T1}), the system enters a pre-alarm state, suspending new
take-offs or landings while allowing those already in progress to complete.  
If the temperature rises further above the higher threshold (\texttt{Temp2}) and
remains there for more than \texttt{T2}, the system switches to alarm mode: the hangar
door closes automatically, the alarm indicator is activated, and all operations are
halted until the user presses the \textbf{RESET} button.  
Through this supervision, the SystemTask guarantees that all flight operations take
place only under safe and stable environmental conditions.

\vspace{\baselineskip}
\section{Coordination of Operations}
When the system starts, the \textbf{FlightTask} determines its initial state from the
Context.  
If all operations are completed, it starts in the \texttt{IDLE} state; otherwise, it
initializes directly in the \texttt{OPERATING} state.  
This mechanism allows automatic recovery after a power loss or reset, enabling the
system to resume any operation that was already in progress.

During normal execution, the FlightTask monitors Context variables to track the
progress of take-off or landing.  
It remains in the \texttt{OPERATING} state while subordinate actions — such as door
movement or landing detection — are active, and returns to \texttt{IDLE} only once all
tasks have completed.  
This ensures proper synchronization between software logic and the physical behaviour
of the hangar.

\vspace{\baselineskip}
\section{Supporting Tasks and Context Interaction}
Several supporting tasks operate concurrently to maintain overall responsiveness:
\begin{itemize}
  \item The \textbf{GateTask} drives the servo motor to open or close the hangar door.
  \item The \textbf{BlinkTask} controls the status LEDs, providing visual feedback
        during active flight phases.
  \item The \textbf{TakeoffTask} and \textbf{LandingTask} monitor distance sensors and
        determine when the drone has completely left or entered the hangar.
  \item The \textbf{ObserverTask} continuously monitors sensors and communication,
        updating the Context with temperature, distance, and command data.
\end{itemize}

All these components interact exclusively through the Context, which acts as a shared
repository of information and coordination.  
For instance, the SystemTask may block new flights by setting a flag that the
FlightTask interprets as a constraint, while the ObserverTask updates the Context
when a new landing command is received, prompting the FlightTask to react in the next
cycle.

\vspace{\baselineskip}
\section{Global Workflow}
The complete operation of the Smart Drone Hangar results from the cooperation of all
tasks under the control of the scheduler.  
The \textbf{ObserverTask} continuously updates the Context with sensor data and
commands received from the drone interface, providing the information base for the
entire system.  
The \textbf{SystemTask} supervises the hangar’s safety state, authorizing or blocking
flight operations according to temperature conditions.  
When flight is allowed and a valid command is received, the \textbf{FlightTask}
coordinates the corresponding sequence, activating the \textbf{GateTask} and
\textbf{BlinkTask} to manage physical actions and visual indicators, while delegating
monitoring to the \textbf{TakeoffTask} or \textbf{LandingTask}.  
Once the sequence concludes, the system automatically returns to its standby state,
ready for the next command.

\vspace{\baselineskip}
Through this cooperative workflow, the Smart Drone Hangar behaves as an integrated
and deterministic system: each task focuses on its own responsibility while sharing
information through the Context.  
This structure guarantees safety, modularity, and continuous operation even in case of
temporary interruptions or environmental changes.

\appendix
\chapter{User Interface}

The user interface of the application is organized into three main panels,
each serving a distinct purpose within the system.

\section{Connection Panel}
The upper part of the view is dedicated to managing the connection
with the serial
port. The user can select the desired port, configure the
communication parameters,
and establish or terminate the connection with the device.

\section{Control Panel}
The central part of the interface provides simple controls for
operating the drone.
It includes two buttons — \textit{Takeoff} and \textit{Landing} —
which send the corresponding
commands to the controller. These buttons are automatically enabled or disabled
based on the current connection status and system state.

\section{Status Panel}
The lower part of the view displays real-time information about the
\textit{Hangar} and \textit{Drone} states, allowing continuous monitoring of the
overall system status.

\begin{figure}[H]
  \centering
  \includegraphics[width=\textwidth]{./img/app.png}
  \caption{App view}
  \label{fig:app}
\end{figure}

\chapter{User Guide}

\section{Cloning the repository}
Clone the project from GitHub and change to the project directory:
\begin{Verbatim}[tabsize=4,xleftmargin=2em]
> git clone https://github.com/alessandrorebosio/ESIOT25.git
> cd ESIOT25/assignment-02
\end{Verbatim}

\section{Connecting and starting the application}
Connect the Arduino to your computer, identify the serial port, then
start the application with:
\begin{Verbatim}[tabsize=4,xleftmargin=2em]
> java -jar drone-hangar-unit-all.jar
\end{Verbatim}

\vspace{\baselineskip}
\noindent Alternatively you can start the project with:
\begin{Verbatim}[tabsize=4,xleftmargin=2em]
> ./gradlew run
\end{Verbatim}

\end{document}
