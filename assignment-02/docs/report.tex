\documentclass[12pt,a4paper]{report}

\usepackage{alltt, fancyvrb, url}
\usepackage{graphicx}
\usepackage[utf8]{inputenc}
\usepackage{float}
\usepackage{xcolor}
\usepackage{hyperref}
\usepackage{longtable}
\usepackage{listings}
\usepackage{color}
\usepackage[utf8]{inputenc}

\usepackage{enumitem}
\usepackage{geometry}
\usepackage{pdfpages}

\geometry{margin=1in}

\definecolor{codegray}{rgb}{0.5,0.5,0.5}
\definecolor{codepurple}{rgb}{0.58,0,0.82}
\definecolor{backcolour}{rgb}{0.95,0.95,0.92}

\lstdefinestyle{sqlstyle}{
  language=SQL,
  backgroundcolor=\color{backcolour},
  commentstyle=\color{codegreen},
  keywordstyle=\color{codepurple},
  numberstyle=\numberstyle,
  stringstyle=\color{codepurple},
  basicstyle=\footnotesize\ttfamily,
  breakatwhitespace=false,
  breaklines=true,
  captionpos=b,
  keepspaces=true,
  numbers=left,
  numbersep=10pt,
  showspaces=false,
  showstringspaces=false,
  showtabs=false
}
\newcommand\numberstyle[1]{%
  \footnotesize
  \color{codegray}%
  \ttfamily
  \ifnum#1<10 0\fi#1 |%
}

\usepackage{textcomp}
\usepackage{siunitx}

\usepackage[english]{babel}
\usepackage[capitalise, english]{cleveref}

\graphicspath{ {./src/img} }

\textwidth=450pt\oddsidemargin=0pt
\begin{document}

\begin{titlepage}
  \begin{center}
    {{\Large{\textsc{Alma Mater Studiorum $\cdot$ University of
    Bologna}}}}\\[1ex]
    \rule{15.8cm}{0.6mm}\\[2ex]
    {\small Degree Programme in Engineering and Computer Science}\\
    {\small{ A.A. 2025/26}}
  \end{center}

  \vfill
  \begin{center}
    {\LARGE{\bf Smart Drone Hangar}}\\[1ex]
  \end{center}
  \vfill

  \begin{center}
    \begin{tabular}{@{}c@{\qquad}c@{}}
      {\large\bf Grazia Bochdanovits de Kavna} & {\large\bf
      Alessandro Rebosio} \\[0.5ex]
      {\small matr.\ 0000000000} & {\small matr.\ 0001130557}
    \end{tabular}
  \end{center}

  \vspace{2mm}
  \noindent\rule{\linewidth}{0.4pt}

\end{titlepage}

\tableofcontents

\chapter{Analysis}

\section{Introduction}
Initially the system starts with the hangar door HD closed; the DRONE
INSIDE state holds, L1 on, L2 and L3 off, and the LCD shows DRONE INSIDE.

\vspace{\baselineskip}
Take-off: the drone requests opening via DRU. On command the HD
opens, the LCD shows TAKE OFF, and the system waits for exit. Exit is
detected by the DDD: when distance {$>$} D1 for more than T1 the drone is
assumed out, the  HD closes and the LCD shows DRONE OUT.

\vspace{\baselineskip}
Landing: the drone requests opening via DRU. If DPD detects presence,
the HD opens and the LCD shows LANDING. When DDD measures distance {$ < $}
D2 for more than T2 the drone is landed, the HD closes and the LCD
shows DRONE INSIDE.

\vspace{\baselineskip}
During take-off/landing L2 blinks (0.5 s period); otherwise it is off.

\vspace{\baselineskip}
Temperature monitoring runs whenever the drone is inside (rest,
take-off, landing). If temperature $\geq$ Temp1 for more than T3 the
system enters pre-alarm: new take-offs/landings are suspended until
return to normal operation (in-progress operations may complete). If
temperature $\geq$ Temp2 ($>$ Temp1) for more than T4 the HD is
closed (if open), L3 turns on and the LCD shows ALARM. If the drone
is outside, an ALARM message is sent via DRU. All operations stay
suspended until the RESET button is pressed; pressing it returns the
system to normal operation.

\vspace{\baselineskip}
\noindent Parameters D1, D2, T1, T2, T3, T4, Temp1, Temp2 are left
configurable for testing.

The DRU GUI must allow:
\begin{itemize}
  \item sending take-off/landing commands (simulate the drone);
  \item displaying drone state (rest, taking off, operating, landing);
  \item displaying hangar state (normal, ALARM);
  \item (during landing) showing current distance to ground.
\end{itemize}

\end{document}
