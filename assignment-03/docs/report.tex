\documentclass[12pt,a4paper]{report}

\usepackage{alltt, fancyvrb, url}
\usepackage{graphicx}
\usepackage[utf8]{inputenc}
\usepackage{float}
\usepackage{xcolor}
\usepackage{hyperref}
\usepackage{longtable}
\usepackage{listings}
\usepackage{color}
\usepackage[utf8]{inputenc}

\usepackage{enumitem}
\usepackage{geometry}
\usepackage{pdfpages}

\geometry{margin=1in}

\usepackage{textcomp}
\usepackage{siunitx}

\usepackage[english]{babel}
\usepackage[capitalise, english]{cleveref}

\graphicspath{ {./src/img} }

\textwidth=450pt\oddsidemargin=0pt
\begin{document}

\begin{titlepage}
  \begin{center}
    {{\Large{\textsc{Alma Mater Studiorum $\cdot$ University of
    Bologna}}}}\\[1ex]
    \rule{15.8cm}{0.6mm}\\[2ex]
    {\small Bachelor's Degree in Computer Engineering and Science}\\
    {\small{ A.A. 2025/26}}
  \end{center}

  \vfill
  \begin{center}
    {\LARGE{\bf Smart Tank Monitoring System}}\\[1ex]
  \end{center}
  \vfill

  \begin{center}
    \begin{tabular}{@{}c@{\qquad}c@{}}
      {\large\bf Grazia Bochdanovits de Kavna} & {\large\bf
      Alessandro Rebosio} \\[0.5ex]
      {\small matr.\ 0001117082} & {\small matr.\ 0001130557}
    \end{tabular}
  \end{center}

  \vspace{2mm}
  \noindent\rule{\linewidth}{0.4pt}

\end{titlepage}

\tableofcontents

\chapter{Analysis}

This chapter analyses the functional behaviour and hardware structure of the Smart Tank Monitoring System, outlining its operating sequences, safety logic, and implementation requirements.  
The objective is to define how the system reacts to flight commands and environmental conditions, and how each hardware element contributes to its autonomous operation.

\section{Description and Requirements}
\section{Summary}

\chapter{Architecture}

\chapter{Arduino Workflow}

\section{Overview}

\section{System Dynamics}

\section{Coordination of Operations}

\section{Supporting Tasks and Context Interaction}

\section{Global Workflow}

\appendix
\chapter{User Interface}

\chapter{User Guide}

\section{Setting up the Python environment}
Navigate to the control unit subsystem directory and create a Python virtual environment:
\begin{Verbatim}[tabsize=4,xleftmargin=2em]
> cd control-unit-subsystem
> python3 -m venv venv
\end{Verbatim}

\noindent Activate the virtual environment:
\begin{itemize}
\item On macOS/Linux:
\begin{Verbatim}[tabsize=4,xleftmargin=2em]
> source venv/bin/activate
\end{Verbatim}
\item On Windows:
\begin{Verbatim}[tabsize=4,xleftmargin=2em]
> venv\Scripts\activate
\end{Verbatim}
\end{itemize}

\noindent Install the required dependencies:
\begin{Verbatim}[tabsize=4,xleftmargin=2em]
> pip install -r requirements.txt
\end{Verbatim}

\section{Flashing the embedded devices}
The system requires flashing firmware to both the ESP32 and Arduino boards.

\subsection{Tank Monitoring Subsystem (ESP32)}
Navigate to the tank monitoring subsystem and upload the firmware:
\begin{Verbatim}[tabsize=4,xleftmargin=2em]
> cd ../tank-monitoring-subsystem
> platformio run --target upload
\end{Verbatim}

\subsection{Water Channel Subsystem (Arduino)}
Navigate to the water channel subsystem and upload the firmware:
\begin{Verbatim}[tabsize=4,xleftmargin=2em]
> cd ../water-channel-subsystem
> platformio run --target upload
\end{Verbatim}

\section{Running the project}
After setting up the Python environment and flashing the embedded devices, you can start the Smart Tank Monitoring System.

\subsection{Starting the Control Unit}
From the control-unit-subsystem directory (with the virtual environment activated), run:
\begin{Verbatim}[tabsize=4,xleftmargin=2em]
> python control-unit-subsystem/main.py
\end{Verbatim}

\subsection{Starting the Dashboard}
Open the dashboard in your web browser:
\begin{Verbatim}[tabsize=4,xleftmargin=2em]
> open ../dashboard-subsystem/index.html
\end{Verbatim}

\noindent Or navigate to the file location and open \texttt{index.html} in your preferred web browser.

\end{document}
