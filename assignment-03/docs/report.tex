\documentclass[12pt,a4paper]{report}

\usepackage{alltt, fancyvrb, url}
\usepackage{graphicx}
\usepackage[utf8]{inputenc}
\usepackage{float}
\usepackage{xcolor}
\usepackage{hyperref}
\usepackage{longtable}
\usepackage{listings}
\usepackage{color}
\usepackage[utf8]{inputenc}

\usepackage{enumitem}
\usepackage{geometry}
\usepackage{pdfpages}

\geometry{margin=1in}

\usepackage{textcomp}
\usepackage{siunitx}

\usepackage[english]{babel}
\usepackage[capitalise, english]{cleveref}

\graphicspath{ {./src/img} }

\textwidth=450pt\oddsidemargin=0pt
\begin{document}

\begin{titlepage}
  \begin{center}
    {{\Large{\textsc{Alma Mater Studiorum $\cdot$ University of
    Bologna}}}}\\[1ex]
    \rule{15.8cm}{0.6mm}\\[2ex]
    {\small Bachelor's Degree in Computer Engineering and Science}\\
    {\small{ A.A. 2025/26}}
  \end{center}

  \vfill
  \begin{center}
    {\LARGE{\bf Smart Tank Monitoring System}}\\[1ex]
  \end{center}
  \vfill

  \begin{center}
    \begin{tabular}{@{}c@{\qquad}c@{}}
      {\large\bf Grazia Bochdanovits de Kavna} & {\large\bf
      Alessandro Rebosio} \\[0.5ex]
      {\small matr.\ 0001117082} & {\small matr.\ 0001130557}
    \end{tabular}
  \end{center}

  \vspace{2mm}
  \noindent\rule{\linewidth}{0.4pt}

\end{titlepage}

\tableofcontents

\chapter{Analysis}

This chapter analyses the functional behaviour and hardware structure
of the Smart Tank Monitoring System.

\section{Description and Requirements}

The Smart Tank Monitoring System is designed to monitor rainwater
levels in a tank and automatically control the opening of a water
channel based on those levels. The system operates in two modes:
\texttt{AUTOMATIC} and \texttt{MANUAL}, with \texttt{AUTOMATIC} as
the default startup mode.

\vspace{\baselineskip}
\noindent\textbf{Tank Monitoring Subsystem (TMS).}
The TMS is an ESP32-based embedded system responsible for
continuously monitoring the rainwater level using sonar sensors.
Measurements are sampled at a configurable frequency $F$ and
transmitted to the Control Unit Subsystem (CUS) via MQTT. When the
system is operating correctly, the green LED is on and the red LED is
off. If network connectivity is lost, the red LED turns on and the
green LED turns off, signaling a communication failure.

\vspace{\baselineskip}
\noindent\textbf{Water Channel Subsystem (WCS).}
The WCS is an Arduino-based embedded system that controls a water
channel valve through a servo motor. The opening range spans from 0\%
(channel closed, 0 degrees) to 100\% (channel fully open, 90
degrees). The WCS includes a tactile button to switch between
\texttt{AUTOMATIC} and \texttt{MANUAL} modes. In \texttt{MANUAL}
mode, operators use a potentiometer to directly control the valve
opening level. An LCD display shows the current valve opening
percentage and the system mode (\texttt{AUTOMATIC}, \texttt{MANUAL},
or \texttt{UNCONNECTED}).

\vspace{\baselineskip}
\noindent\textbf{Control Unit Subsystem (CUS).}
The CUS is the main back-end system running on a PC that orchestrates
the Smart Tank Monitoring System. It receives rainwater level data
from the TMS via MQTT and implements the control logic. When the
level exceeds $L_1$ for longer than $T_1$, the valve opens to 50\%;
if it reaches $L_2$, the valve opens to 100\%. The CUS communicates
with the WCS via serial connection and provides HTTP endpoints for
the Dashboard Subsystem. If no data is received from the TMS for more
than $T_2$ time units, the system enters \texttt{UNCONNECTED} mode.

\newpage
\vspace{\baselineskip}
\noindent\textbf{Dashboard Subsystem (DBS).}
The DBS is a web-based front-end accessible from any device connected
via HTTP. It displays real-time graphs of the rainwater level over
the last $N$ measurements, the current valve opening percentage, and
the system state (\texttt{MANUAL}, \texttt{AUTOMATIC},
\texttt{UNCONNECTED}, or \texttt{NOT AVAILABLE}). The dashboard
includes GUI controls to switch between \texttt{MANUAL} and
\texttt{AUTOMATIC} modes and a widget to manually adjust the valve
opening level when in \texttt{MANUAL} mode.

\vspace{\baselineskip}
\noindent\textbf{System Architecture.}
The four subsystems interact as follows:
\begin{itemize}
  \item TMS sends rainwater level data to CUS via MQTT.
  \item CUS processes level data, enforces the control policy, and
    commands the WCS via serial communication.
  \item WCS receives commands from CUS and locally manages valve
    control and operator interaction via the potentiometer and button.
  \item DBS queries the CUS via HTTP to display real-time system
    status and provides remote operators with monitoring and control
    capabilities.
\end{itemize}

\section{Summary}

The Smart Tank Monitoring System is a distributed embedded system
that combines automatic and manual control over water channel
management. At startup, the system initializes in \texttt{AUTOMATIC}
mode, with the TMS continuously sampling the rainwater level and
transmitting it to the CUS. The CUS evaluates the level against two
thresholds: if it exceeds $L_1$ for longer than $T_1$, the valve
opens to 50\% to begin draining; if it reaches $L_2$, the valve
immediately opens to 100\%. This dual-threshold approach ensures
proportional response to gradually rising levels while providing
emergency drainage for critical conditions.

Operators can switch to \texttt{MANUAL} mode by pressing the WCS
button, allowing direct control o f the valve through a
potentiometer. The LCD display on the WCS provides real-time feedback
of the current opening percentage and mode.The DBS offers remote
monitoring and control, displaying historical graphs of rainwater
levels and system status.

In case of network failures, the system gracefully degrades to
\texttt{UNCONNECTED} mode, with the TMS signaling the problem via its
red LED and the LCD showing the unavailable state. This design
ensures continuous operation of the WCS in \texttt{MANUAL} mode, even
when the CUS becomes unreachable, maintaining basic functionality
during network outages. Once connectivity is restored, the system
automatically returns to normal operation.

The system achieves modularity through MQTT publish-subscribe
communication, HTTP REST APIs, and serial protocols, allowing each
subsystem to operate independently while coordinating through the
CUS. This architecture supports scalability, fault tolerance, and
ease of maintenance.

\newpage
\section{Wiring}

Table~\ref{tab:tms-hardware} lists the essential hardware components
required to build and test the Tank Monitoring Subsystem (TMS).
Quantities, component types, and their function are summarized for
procurement and wiring reference.

\begin{table}[H]
  \centering
  \small
  \resizebox{\textwidth}{!}{%
    \begin{tabular}{@{}r l p{0.6\textwidth}@{}}
      \hline
      Qty & Component & Notes \\ \hline
      1 & ESP32 SoC board & Handles sensors, LEDs, and MQTT communication \\
      1 & Sonar sensor & Measures rainwater level in the tank \\
      1 & Green LED & Indicates normal operation and network connectivity \\
      1 & Red LED & Signals network problems or communication failures \\
      2 & Resistors 220\,$\Omega$ & Current-limiting resistors for
      LEDs \\ \hline
    \end{tabular}%
  }
  \caption{Essential hardware list for Tank Monitoring Subsystem
  (TMS)}
  \label{tab:tms-hardware}
\end{table}

\noindent Figure~\ref{fig:tms-wiring} shows the wiring diagram of the
TMS, highlighting the main connections between the ESP32, sonar
sensor, and status LEDs.

\begin{figure}[H]
  \centering
  \includegraphics[width=\textwidth]{./img/tms-wiring.png}
  \caption{Wiring diagram of the Tank Monitoring Subsystem (TMS).}
  \label{fig:tms-wiring}
\end{figure}

\noindent\textit{Note: The wiring diagram shows an Arduino UNO for
illustration purposes, but the actual implementation uses an ESP32 SoC board.}

\newpage
\noindent Table~\ref{tab:wcs-hardware} lists the essential hardware
components required to build and test the Water Channel Subsystem
(WCS).

\begin{table}[H]
  \centering
  \small
  \resizebox{\textwidth}{!}{%
    \begin{tabular}{@{}r l p{0.6\textwidth}@{}}
      \hline
      Qty & Component & Notes \\ \hline
      1 & Arduino UNO board & Handles valve control, sensors, and
      serial communication \\
      1 & Servo motor & Controls the water channel valve opening
      (0--90 degrees) \\
      1 & Potentiometer & Allows manual control of valve opening in
      MANUAL mode \\
      1 & Tactile button & Switches between AUTOMATIC and MANUAL modes \\
      1 & LCD display & Displays valve opening percentage and system mode \\
      1 & Resistor 10\,k$\Omega$ & Pull-up / pull-down resistor for
      the button \\ \hline
    \end{tabular}%
  }
  \caption{Essential hardware list for Water Channel Subsystem (WCS)}
  \label{tab:wcs-hardware}
\end{table}

\noindent Figure~\ref{fig:wcs-wiring} shows the wiring diagram of the
WCS, highlighting the main connections between the Arduino, servo
motor, potentiometer, button, and LCD display.

\begin{figure}[H]
  \centering
  \includegraphics[width=\textwidth]{./img/wcs-wiring.png}
  \caption{Wiring diagram of the Water Channel Subsystem (WCS).}
  \label{fig:wcs-wiring}
\end{figure}

\chapter{Architecture}

\chapter{Arduino Workflow}

\section{Overview}

\section{System Dynamics}

\section{Coordination of Operations}

\section{Supporting Tasks and Context Interaction}

\section{Global Workflow}

\appendix
\chapter{User Interface}
\section{Overview}

The user interface consists of a web-based dashboard that provides
real-time monitoring and control of the Smart Tank Monitoring System.
Through this interface, you can visualize system data, switch between
automatic and manual operation modes, and manually control the valve
when needed.

\begin{figure}[H]
  \centering
  \includegraphics[width=\textwidth]{./img/app.png}
  \caption{App view}
  \label{fig:app}
\end{figure}

\chapter{User Guide}

\section{Setting up the Python environment}
Navigate to the control unit subsystem directory and create a Python
virtual environment:
\begin{Verbatim}[tabsize=4,xleftmargin=2em]
> cd control-unit-subsystem
> python3 -m venv venv
\end{Verbatim}

\noindent Activate the virtual environment:
\begin{itemize}
  \item On macOS/Linux:
\begin{Verbatim}[tabsize=4,xleftmargin=2em]
> source venv/bin/activate
\end{Verbatim}
  \item On Windows:
\begin{Verbatim}[tabsize=4,xleftmargin=2em]
> venv\Scripts\activate
\end{Verbatim}
\end{itemize}

\noindent Install the required dependencies:
\begin{Verbatim}[tabsize=4,xleftmargin=2em]
> pip install -r requirements.txt
\end{Verbatim}

\section{Flashing the embedded devices}
The system requires flashing firmware to both the ESP32 and Arduino boards.

\subsection{Tank Monitoring Subsystem (ESP32)}
Navigate to the tank monitoring subsystem and upload the firmware:
\begin{Verbatim}[tabsize=4,xleftmargin=2em]
> cd ../tank-monitoring-subsystem
> platformio run --target upload
\end{Verbatim}

\subsection{Water Channel Subsystem (Arduino)}
Navigate to the water channel subsystem and upload the firmware:
\begin{Verbatim}[tabsize=4,xleftmargin=2em]
> cd ../water-channel-subsystem
> platformio run --target upload
\end{Verbatim}

\section{Running the project}
After setting up the Python environment and flashing the embedded
devices, you can start the Smart Tank Monitoring System.

\subsection{Starting the Control Unit}
From the control-unit-subsystem directory (with the virtual
environment activated), run:
\begin{Verbatim}[tabsize=4,xleftmargin=2em]
> python control-unit-subsystem/main.py
\end{Verbatim}

\subsection{Starting the Dashboard}
Open the dashboard in your web browser:
\begin{Verbatim}[tabsize=4,xleftmargin=2em]
> open ../dashboard-subsystem/index.html
\end{Verbatim}

\noindent Or navigate to the file location and open
\texttt{index.html} in your preferred web browser.

\end{document}
